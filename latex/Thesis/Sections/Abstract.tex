\textit{
The presence of backflow along boundaries where a natural boundary condition is prescribed can cause numerical instabilities. Different stabilisation techniques may be used to control these instabilities, however they also have an affect on the accuracy of the solution. These techniques usually have a parameter which can be changed and which affects how strongly the backflow is counteracted. The purpose of this paper will be to outline how these parameters can be automatically chosen such that stability is guaranteed while attempting to minimise the stabilisations affect on accuracy. First a stability criterion is found, and then a simplification is made which allows this criterion to be checked more easily. Then each of the stabilisation methods are considered in-depth in order to ascertain whether they are enough for guaranteeing stability or if they rely on other stabilising terms. A variant of the Tangential Derivative Penalisation method is found to be a good candidate for automation, and an algorithm is presented for that automation.
}