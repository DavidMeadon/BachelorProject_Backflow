\mychapter{5}{Parameter Automation}


\section{Algorithms}

Using the previous results, the stability criterion and the simplification to the eigenvalue problem, algorithms can now be created which will automate the parameter selection.\\
\\
The results did show however that for two of the considered methods, Velocity Penalisation and Tangential Penalisation, the methods were unable to attain the desired stability criterion on their own, so with the current tools used in this paper it is not possible to automate them. The Velocity Penalisation method has a value for its parameter which guarantees stability so it could still be safely usable. The Tangential Penalisation method however does not have that benefit and requires more research into how it can be safely implemented so as to remove the danger of backflow instabilities not being fully stabilised by it. The Tangential Penalisation Max method did however show very promising results and as such will be be discussed here and following that an algorithm for the reduction of the eigenvalue problem discussed in the Automation Prerequisites chapter will be outlined.

\subsection{Tangential Penalisation Max}

The results in the previous section for this method agreed with the theoretical finding that if \mgamma~is taken big enough then stability will be attained. Ideally for this method, the parameter would start at the guaranteed stable value when backflow is detected and then proceed from there, however that greatly depends on the geometry of the simulation as the Poincar\'e Constant is non-trivial to calculate in general. Thus it will be suggested that a first value is a user-given one, however it is recommended that if possible, the initial value is the guaranteed stable value. The algorithm is thus outlined in \autoref{algo:TangMax}.

\begin{algorithm}[t]
\SetAlgoLined

\KwResult{Find \mathm{\gamma_{Next}}, the stable parameter value}

 \eIf{First time-step}{
 initial guess = user input\;
 }{
 initial guess = \mathm{\gamma_{prev}}\;
 }
 \mathm{\gamma_{candidate}} = initial guess\;
 Stability Found = false\;
 \While{True}{
 \If{No Backflow}{
    \mathm{\gamma_{next}} = \mathm{\gamma_{candidate}}\;
    break\;
 }
  Test if criterion attained\;
  \uIf{Criterion Attained}{
  \mathm{\gamma_{candidate}} = reduced \mathm{\gamma_{candidate}}\;
  Stability Found = true\;
  }
  \uElseIf{Criterion Failed and Stability Found == false}{
  \mathm{\gamma_{candidate}} = increased \mathm{\gamma_{candidate}}\;
  }
  \Else{
  \mathm{\gamma_{next}} = \mathm{\gamma_{previous}}\;
  break\;
  }
 }
 \caption{Tangential Penalisation Max}
 \label{algo:TangMax}
\end{algorithm}
% The Algorithm for this method is slightly different because there is no guaranteed stable value for its parameter that can easily be computed. However we do know that if . Thus we can begin the automation at some user given input and then determine if the this is a stable value for \mgamma. If it is stable then attempt to reduce it until instability is detected and then save the previously found \mgamma~value as the current best one to use, while if the user given input is not stable then increase the value of the parameter until stability is reached. This would then be repeated at every time step with the previously found \mgamma~value being used as the initial value for all time steps after the first one as this would reduce the number of computations rather than always beginning at some fixed value at every time step. In algorithmic form:

% \begin{algorithm}[H]
% \SetAlgoLined
% \KwResult{Write here the result }
%  initialization\;
%  \While{While condition}{
%   instructions\;
%   \eIf{condition}{
%   instructions1\;
%   instructions2\;
%   }{
%   instructions3\;
%   }
%  }
%  \caption{Automated Parameter Stabilisation for the Tangential Penalisation Max method}
% \end{algorithm}

\subsection{Eigenvalue problem reduction}

In order to reduce the computational complexity of finding the eigenvalues necessary for checking the stability criterion, the matrix can be reduced. This is done by first considering \mathm{M_{backflow}}, the discretised form of \eqref{eq:backflow} which will be zero almost everywhere except for small sub-matrices where there is actually backflow. \mathm{M_{backflow}} is then normalised into a matrix of zeros and ones, where it is zero if there is no backflow and 1 if there is backflow, denote this matrix \mathm{M_{backflow}^{1\&0's}}. Finally \mathm{M_{backflow}^{1\&0's}} is element-wise multiplied with \mathm{M_{stab}}, the discretisation of \mathm{- B + S} and yields \mathm{M_{stab}^{reduced}}. We then extract the sub-matrices from \mathm{M_{stab}^{reduced}} which are the nonzero square matrices where there is backflow. Finally these will be the reduced matrices over which we can calculate eigenvalues and use to automatically stabilise. An additional benefit of this procedure is that it allows us to stabilise different backflow regions with different parameters depending on the amount of backflow there, rather than the current regime where the entire system uses the same parameter irrespective of the amount of backflow.
