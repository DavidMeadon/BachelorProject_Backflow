% \mychapter{6}{Conclusion}
\mychapter{6}{Conclusion}

\section{Conclusion}
During the course of this paper, we have considered the problem of backflow and whether it is possible to automatically stabilise it such that stability is guaranteed while trying to also improve accuracy. A stability criterion was the first main result obtained which could be used to check whether the the stabilisation being used was indeed preventing the backflow term from causing instabilities, and this was then followed by a small observation which simplified the process of checking this criterion. Furthermore these analytic results were then tested and verified where each of the different stabilisation methods where considered in-depth. The Velocity Penalisation method was found to be unable to stabilise the backflow term by itself when its parameter \mbeta~was taken below its guaranteed stable value of 1. The Tangential Penalisation Max method had much better results where the theoretical result was verified that for a \mgamma~large enough, larger than or equal to the poincar\'e constant of the domain, the stabilisation by itself was able to prevent the backflow term from causing instabilities. Finally the Tangential Penalisation method was tested and found that it was completely unable to prevent the backflow term from causing instabilities, irrespective of the value of \mgamma. These results, especially for the Tangential Penalisation Max are quite positive for the possibility of automating the stabilisation precedure.

\section{Discussion and future research}
As to what could further be researched on this topic, there are other proposed methods for stabilising backflow which could also be investigated to see whether they are viable candidates for automation. Furthermore, it would also be beneficial to try and include the other stabilising terms which counteract backflow such as the viscous term. In general this is more computationally challenging because with the inclusion of the viscous term, the matrix reduction idea can no longer be applied with guaranteed certainty as the viscous term applies to the entire domain, moreover after applying boundary conditions, the discretisation may no longer even be symmetric and in that case the symmetric part of the matrix would have to be extracted and checked against the stability criterion.