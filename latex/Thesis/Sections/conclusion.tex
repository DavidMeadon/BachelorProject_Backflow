% \mychapter{6}{Conclusion}
\mychapter{6}{Conclusion}

\section{Conclusion}
During the course of this paper, the problem of backflow instability was considered and whether it is possible to automatically stabilise it such that stability is guaranteed while trying to also improve accuracy. A stability criterion was the first main result obtained which could be used to check whether the the stabilisation being used was indeed preventing the backflow term from causing instabilities, and this was then followed by a small observation which simplified the process of checking this criterion. Furthermore these analytic results were then tested and verified where each of the different stabilisation methods where considered in-depth. The Velocity Penalisation method was found to be unable to stabilise the backflow term by itself when its parameter \mbeta~was taken below its guaranteed stable value of 1. For the Tangential Penalisation Max method, the theoretical stability analysis was verified that for a \mgamma~large enough, the stabilisation by itself was able to prevent the backflow term from causing instabilities. Finally the spectral analysis in the analysed numerical examples showed that the Tangential Penalisation Method does not lead to a full stabilisation of the backflow spectrum. This is consistent with numerical results using high Reynolds numbers. These results, especially for the Tangential Penalisation Max are quite positive for the possibility of automating the stabilisation procedure.

\section{Discussion and future research}
As to what could further be researched on this topic, there are other proposed methods for stabilising backflow which could also be investigated to see whether they are viable candidates for automation. The results from the testing also showed that 2 of the considered methods were unable to stabilise the backflow instabilities on their own, while improving accuracy, and so it would also be beneficial to try and include the other stabilising terms which counteract backflow such as the viscous term observed in \autoref{eq:stabfunc}. In general this is more computationally challenging because with the inclusion of the viscous term, the matrix reduction idea can no longer be applied with guaranteed certainty as the viscous term applies to the entire domain, moreover after applying boundary conditions, the discretisation may no longer even be symmetric and in that case the symmetric part of the matrix would have to be extracted and checked against the stability criterion. Furthermore for the Tangential Penalisation method, we ran into the problem that there was always a single eigenvalue which would not become positive. As said above, perhaps other stabilising terms could be included but since it is always a single eigenvalue, it may be beneficial to consider other techniques such as pole placement, or perhaps using a different stabilisation method, one guaranteeing stability, for that time-step in order to ensure that the eigenvalue becomes positive.