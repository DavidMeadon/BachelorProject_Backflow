% \mychapter{1}{Introduction}
\mychapter{1}{Introduction}


It is often the case in computational fluid dynamics that the velocity of the flow is not prescribed along all boundaries, and instead a natural boundary condition is used, where natural boundary conditions are those that are satisfied after a solution has been found. These boundary conditions can very often occur in practice and are particularly relevant in simulations of physiological flow such as blood moving through veins and arteries or air through the lungs \cite{bertoglio2017}, since in these cases the velocity of the flow on all boundaries can be hard to measure and rather the only available data is a combination of pressure and/or averaged flow. In general these biological systems being modelled may be exceptionally complex so in order to reduce the computational cost of finding a complete solution, the region of interest is usually truncated. Special care must then be taken to ensure the correct boundary conditions are prescribed along the boundaries of these truncated regions, especially where natural boundary conditions must be specified. Along these boundaries where natural boundary conditions occur, the biological systems discussed earlier may experience a reversal of flow due to periodicity in the physiology of the systems and the physical regularity of the structure. This situation of flow reversal is known as backflow.
\\
\\
Backflow, when occuring along boundaries with natural boundary conditions, has the unfortunate consequence of causing numerical instabilities, and as such requires a stabilisation method to be used. A common trend among some of these methods is that they rely on a free parameter which is selected by the the person performing the simulation. While there are values for some of these parameters for which stability has been proven, it has also be observed experimentally that there are values for these parameters where the solution is stable and better accuracy is attained \cite{bertoglio2014}.\\
\\
Since the parameter must be chosen by the user before computation of the solution begins, the problem arises that if the parameter is chosen incorrectly, i.e. the solution becomes unstable, then the entire computation would have to be redone with a different parameter wasting all the previous computation time. How then should this parameter be chosen such that stability is maintained while improving the accuracy of the solution? Thus the focus in this paper will be to investigate how the free parameter in two of these stabilisation method, Velocity Penalisation and Tangential Derivative Penalisation, can be chosen automatically during the computation such that stability is maintained while getting as accurate a solution as possible. This then leads to the following research question: "Can these parameters be automatically chosen either at the beginning of- or during the computation such that stability is maintained while possibly improving the accuracy of the solution".\\
\\
The approach that will be taken will be to investigate how stability is dependent on the parameters and if it is possible to find some form of stability criterion against which the parameters can be checked to see whether backflow instabilities could occur. This would then be a key step in the automation process as once a sufficient condition for stability is found, this condition can be used during the automated parameter selection procedure.\\
\\
In the first chapter, a background of backflow and backflow stabilisation will be given, where an important result necessary for the automation process will be derived. The necessary tools for the automation process, a stability criterion and a way to feasibly attain it will then be detailed in the second chapter. The third chapter will focus on the results of testing our analytical results derived in the second chapter on a test case. The fourth chapter will then detail the suggested algorithms that could be used to automate the parameter selection, and how to exactly reduce the problem such that it does not become too computationally expensive. Finally the report will then be concluded.