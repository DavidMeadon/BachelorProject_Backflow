% \mychapter{1}{Introduction}
\mychapter{1}{Introduction}


It is often the case in computational fluid dynamics that the velocity of the flow is not prescribed along all boundaries, and instead a natural boundary condition is used, where natural boundary conditions are those that are satisfied after a solution has been found. These boundary conditions can very often occur in practice and are particularly relevant in simulations of physiological flow such as blood moving through veins and arteries or air through the lungs \cite{bertoglio2017}, since in these cases the velocity of the flow on all boundaries can be hard to measure and rather the only available data is a combination of pressure and/or averaged flow. In general these biological systems being modelled may be exceptionally complex so in order to reduce the computational cost of finding a complete solution, the region of interest is usually truncated. Special care must then be taken to ensure the correct boundary conditions are prescribed along the boundaries of these truncated regions, especially where natural boundary conditions must be specified. Along these boundaries where natural boundary conditions occur, the biological systems discussed earlier may experience a reversal of flow due to periodicity in the physiology of the systems and the physical regularity of the structure. This situation of flow reversal is known as backflow.



\todo[inline]{Add much more stuff here for the intro}
