% \mychapter{2}{Automating the parameter}
\mychapter{3}{Automation Prerequisites} %Chapter title needs work

The first steps that will be taken in order to understand how the parameters could be automated will be to obtain some tools which will be used when we would want to automate the parameter selection. Understanding exactly how stability can be achieved by using the stabilisations is exceptionally important for the automation so we will first try to find some stability criterion that can be checked during the parameter selection. Following that we will offer an insight that will reduce the computational cost of checking this criterion such that is becomes feasible. Thus, we can now begin with an in-depth look into how a stability criterion can be found.

\section{Stability criterion}
% \noindent\rule{\textwidth}{1pt}
In order to automate the parameters used in the stabilisation process, we first need some stability criterion that we can check and use to ensure that the parameter we are using is indeed guaranteeing stability. To find this criterion we first consider the change of energy of the system. We require the that the energy of the system does not increase and to this end consider \autoref{NSeqENergy}, which is the change in energy over time of the system, and for stability we would like this to be negative, i.e. we want it to be dissipative. This is clearly a problem in the presence of backflow where it may become positive thus causing an instability. To see this more clearly, we use that any real-valued function can be split into its negative and positive part: \( f = \abs{f}_+ - \abs{f}_-\) where \( \abs{f}_+ = \frac{\abs{f} + f}{2}\) and \( \abs{f}_- = \frac{\abs{f} - f}{2}\). Thus \autoref{NSeqENergy} can be rewritten as:

%  and thus we would like in the discretised form that the problem is positive definite. We require the eigenvalues in the direction of the solution to be positive. We do not know though what the direction of the solution is before we have found the solution, so a sufficient condition is to make all the eigenvalues positive.
\[
\begin{aligned}
   \partial_t\frac{\rho}{2}\fint\norm{\bmu}^2 &= -2\mu\fint\norm{\varepsilon(\bmu)}^2 - \frac{\rho}{2}\bint\bmu\cdot\bmn\norm{\bmu}^2\\
                                              &= -\qty(2\mu\fint\norm{\varepsilon(\bmu)}^2 + \frac{\rho}{2}\bint\bmu\cdot\bmn\norm{\bmu}^2)\\
                                              &= -\qty(2\mu\fint\norm{\varepsilon(\bmu)}^2 + \frac{\rho}{2}\bint\abs{\bmu\cdot\bmn}_+\norm{\bmu}^2 - \frac{\rho}{2}\bint\abs{\bmu\cdot\bmn}_-\norm{\bmu}^2)
\end{aligned}
\]
where now, we could like the term between brackets:
\begin{equation}\label{eq:instabfunc}
    2\mu\fint\norm{\varepsilon(\bmu)}^2 + \frac{\rho}{2}\bint\abs{\bmu\cdot\bmn}_+\norm{\bmu}^2 - \frac{\rho}{2}\bint\abs{\bmu\cdot\bmn}_-\norm{\bmu}^2
\end{equation}
to be positive. This is obviously not always true due to the last term in \autoref{eq:instabfunc} which is exactly the term that will become non-zero in the presence of backflow. Furthermore from hereon we will denote:
\begin{equation}\label{eq:backflow}
    B := \frac{\rho}{2}\bint\abs{\bmu\cdot\bmn}_- \bmu \cdot \bmv
\end{equation} as the backflow term since it will be equal to the last term in \autoref{eq:stabfunc} when testing with the true solution, \mathm{\bmu}. So we now have the beginning of our stability criterion, that is that we require \autoref{eq:instabfunc} to be positive. As this is not true in general, we need to add one of the before mentioned stabilising terms to ensure positivity. Consider now \autoref{eq:NSweakstab} and take (\bmv~= \bmu), i.e. test diagonally, then after a similar derivation as in the background chapter, one would attain a similar equation as \autoref{NSeqENergy} with an additional term corresponding to the stabilisation. With a similar argument as above, the terms of interest thus become:
\begin{equation}\label{eq:stabfunc}
    2\mu\fint\norm{\varepsilon(\bmu)}^2 + \frac{\rho}{2}\bint\abs{\bmu\cdot\bmn}_+\norm{\bmu}^2 - \frac{\rho}{2}\bint\abs{\bmu\cdot\bmn}_-\norm{\bmu}^2 + S_{|\bmv=\bmu}
\end{equation}
where \mathm{S_{|\bmv=\bmu}} is one of the stabilisation methods tested at the true solution of \bmu. Thus the condition we would like to check for stability is if the eigenvalues of \autoref{eq:NSweakstab} in the direction of the solution, \bmu, are positive. Here we will take a conservative view and require this condition only for \mathm{- B + S} since if \autoref{eq:NSweakstab} is rewritten, \mathm{-B} is the only term which will contribute to the backflow instabilities. It is in general not feasible to look at the eigenvalues in the direction of the solution because we will not know the solution before we compute it, so a sufficient condition will be to ensure that all the eigenvalues are positive. This now is the stability criterion which we will use to ascertain whether the stabilisation method is indeed stabilising the backflow term, and which will be checked during the automation procedure:
\theoremstyle{definition}
\begin{criterion}
The numerical simulation will not suffer from backflow instabilities if the eigenvalues of the discretisation of \mathm{- B + S} are non-negative.
\end{criterion}
For a general problem however, this criterion would be hard to check as the size of the discretised matrix will be exceptionally large, and thus we would have to calculate many eigenvalues at every time-step in order to set the value for the parameter. We can however cheapen this significantly by using certain properties of the matrix.


\section{Reduction of eigenvalue problem}

Finding the eigenvalues of the entire backflow term's matrix will in general be exceptionally expensive due to the size of the matrix, however we can reduce the problem by first noting that the discretisation of the backflow term, \mathm{B}, will be zero everywhere except where backflow occurs, namely where the instability problems occur. Thus for stability, we are only concerned that the regions where backflow is occurring are being stabilised. So we can therefore reduce the problem of finding the eigenvalues of \mathm{- B + S} to only considering the sub-matrices of that discretisation where backflow actually occurs. An algorithm outlining exactly how the matrix is reduced can be found in the Parameter Automation chapter.\\
\\
We now have some tools which will allow us to automate the parameter selection, and so we can now test these tools in the next chapter on a test case, and thereby observe if the behaviour of the stabilisation methods is as expected, and if it is indeed possible to automate all the methods.