%% Useful packages

%% Language and font encodings
\usepackage[english]{babel}
\usepackage[utf8]{inputenc}
\usepackage[T1]{fontenc}
\usepackage{amsfonts}
\usepackage{bm}

%% Sets page size and margins
% \usepackage[a4paper,top=1in,bottom=1in,left=1.5in,right=1in]{geometry}
\usepackage[a4paper,width=150mm,top=25mm,bottom=25mm,bindingoffset=6mm]{geometry}
\usepackage[titletoc]{appendix}
\usepackage{fancyhdr}
\pagestyle{fancy}
\setlength{\headheight}{14.5pt}
\renewcommand{\chaptermark}[1]{\markboth{#1}{}}
\fancyhf{}
\fancyhead[RO,LE]{\leftmark}
% \fancyhead[RO,LE]{Chapter \thechapter}
% \fancyhead{}
% \fancyhead[RO,LE]{Automation of Backflow Stabilisation Parameters}
% \fancyhead[RO,LE]{Chapter \thechapter: }
% \fancyhead[RE,LO]{\leftmark}
\fancyfoot{}
\fancyfoot[CE,CO]{\thepage}
% \fancyfoot[CE,CO]{Chapter \thechapter}
% \fancyfoot[CO,RE]{Author Name}

%% Math
\usepackage{amsmath} 
\usepackage{amssymb}
\usepackage{amsthm}
\usepackage{mathtools}
\usepackage{mathpazo}
\usepackage{physics}
\usepackage{dsfont} %% Nice double stroke letters like R for real numbers
% \usepackage{siunitx} %%Adding SI units

%% Graphics
\usepackage{graphicx}
\graphicspath{ {media/} }
\usepackage{xcolor}

%% Figures
\usepackage{subcaption}
% \usepackage{wrapfig}
% \usepackage{float}
\usepackage{caption}

%% Tables
% \usepackage{csvsimple}
% \usepackage{multicol}
% \usepackage{multirow}

%% Adding boxes around text
% \usepackage{fancybox, calc}

%% Format
\usepackage[colorinlistoftodos]{todonotes}
\usepackage[colorlinks=true, allcolors=blue]{hyperref}
\usepackage{enumerate}
\usepackage{csquotes}
\usepackage[backend=biber,style=ieee,url=true,doi=true,dashed=false,sorting=nyt]{biblatex}
\addbibresource{citations.bib}
\usepackage{comment}
\usepackage{bookmark} %For correct pdf bookmarks

%% For Title Page
\usepackage{pdfpages}

%% For Code
% \usepackage{listings}
% \usepackage{minted}

%% Temp Text
% \usepackage{lipsum}

%% For algorithms
\usepackage[boxed]{algorithm2e}


% %%%%%%%%%%%%--FOR ABS AND NORM--%%%%%%%%%%%%
%% Not needed when using physics package
% \DeclarePairedDelimiter\abs{\lvert}{\rvert}
% \DeclarePairedDelimiter\norm{\lVert}{\rVert}

% \makeatletter
% \let\oldabs\abs
% \def\abs{\@ifstar{\oldabs}{\oldabs*}}
% %
% \let\oldnorm\norm
% \def\norm{\@ifstar{\oldnorm}{\oldnorm*}}
% \makeatother
% %%%%%%%%%%%%%%%%%%%%%%%%%%%%%%%%%%%%%%%%%%%%



%%%%%%%%%%%%%%%%%%%%%%%%%%%%%%%%--OWN COMMANDS--%%%%%%%%%%%%%%%%%%%%%%%%%%%%%%%%

%%For Criterion Environment
\theoremstyle{plain}
\newtheorem*{criterion}{Stability Criterion}

%% Mini Mathmode
\newcommand{\mathm}[1]{\(#1\)}

%% For Bold text u, v, n
\newcommand{\bmu}{{\textbf{u}}}
\newcommand{\bmv}{{\textbf{v}}}
\newcommand{\bmn}{{\textbf{n}}}

%% For domain(f) and boundary(b) integrals 
\newcommand{\bint}{\int\limits_{\Gamma_{N}}}
\newcommand{\fint}{\int\limits_{\Omega}}

%% For making own chapters without numbering
%% in the table of contents while still
%% correct section numbering
\newcommand{\mychapter}[2]{
    \markboth{#2}{}
    \setcounter{chapter}{#1}
    \setcounter{section}{0}
    \chapter*{#2}
    \addcontentsline{toc}{chapter}{#2}}
    
%% Creates an "order of" symbol
\DeclareRobustCommand{\orderof}{\ensuremath{\mathcal{O}}}

%% For inline math beta and gamma
\newcommand{\mbeta}{\(\beta\)}
\newcommand{\mgamma}{\(\gamma\)}

%% For inner products
\newcommand{\inner}[2]{\big(#1,#2\big)}

%% For stress tensor
\newcommand{\stresstens}{\bm{\varepsilon}(\bmu)}


%% This is for simpler integrals

% \usepackage{xparse}

% \newcommand\dd{\mathrm{d}}

% \ExplSyntaxOn
% \NewDocumentCommand \Int { s o m o }
%   {%
%     \IfNoValueTF{ #2 }
%       { \int }
%       {
%         \fiziks_int:nn { #1 } { #2 }
%       }
%     #3
%     \IfNoValueF { #4 } { \fiziks_int_dx:n { #4 } }
%   }

% \seq_new:N \l__fiziks_int_args_seq
% \tl_new:N \l__fiziks_int_ast_tl

% \cs_new_protected:Npn \fiziks_int:nn #1 #2
%   {
%     \seq_set_split:Nnn \l__fiziks_int_args_seq { ; } { #2 }
%     \seq_map_inline:Nn \l__fiziks_int_args_seq 
%       {
%         \tl_if_in:nnTF { ##1 } { * }
%           {% * case
%             \tl_set:Nn \l__fiziks_int_ast_tl { ##1 }
%             \tl_remove_once:Nn \l__fiziks_int_ast_tl { * }
%             \fiziks_int_inner:NnV \oint { #1 } \l__fiziks_int_ast_tl
%           }
%           {% no * case
%             \fiziks_int_inner:Nnn \int { #1 } { ##1 }
%           }
%       }
%   }
% \cs_new_protected:Npn \fiziks_int_inner:Nnn #1 #2 #3
%   {
%     #1
%     \tl_if_blank:nF { #3 } 
%       {
%         \IfBooleanT { #2 } { \limits }
%         \fiziks_int_inner_inner:Nx \sb { \clist_item:nn { #3 } { 1 } }
%         \fiziks_int_inner_inner:Nx \sp { \clist_item:nn { #3 } { 2 } }
%       }
%   }
% \cs_generate_variant:Nn \fiziks_int_inner:Nnn { NnV }
% \cs_new:Npn \fiziks_int_inner_inner:Nn #1 #2
%   {
%     \tl_if_blank:nF { #2 } { #1 { #2 } }
%   }
% \cs_generate_variant:Nn \fiziks_int_inner_inner:Nn { Nx }

% \cs_new_protected:Npn \fiziks_int_dx:n #1
%   {
%     \seq_set_split:Nnn \l__fiziks_int_args_seq { ; } { #1 }
%     \seq_map_inline:Nn \l__fiziks_int_args_seq
%       {
%         \,\dd##1
%       }
%   }
% \ExplSyntaxOff
% %%      Can be used as \Int[0,T;0,a]{f^2}[t;x]    
%%%%%%%%%%%%%%%%%%%%%%%%%%%%%%%%%%%%%%%%%%%%%%%%%%%%%%%%%%%%%%%%%%%%%%%%%%%%%%%%




% \addto\extrasngerman{%
%   \def\figureautorefname{\textit{Bild}} %
%   \def\tableautorefname{\textit{Tabelle}}%
%   \def\subsectionautorefname{Unterkapitel}%
%   \def\sectionautorefname{Kapitel}%
%   \def\subsubsectionautorefname{Abschnitt}%
%   \def\equationautorefname{\textit{Gleichung}}%
% }
