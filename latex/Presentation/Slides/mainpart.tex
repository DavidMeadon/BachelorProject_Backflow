\section{Automation Prerequisites}

 \subsection{What are we investigating?}
    \begin{frame} {What are we investigating}
        
        \begin{block}{What is needed for automating the parameters?}
        \begin{itemize}
            \item A criterion to be met
            \item A way to test and attain that criterion
        \end{itemize}
        \end{block}
    \end{frame}
  
    \subsection{Stability Criterion}
    \begin{frame}{What do we want from the system}
        Would like the system to be dissipative.\\
        Change in energy:
        \[\footnotesize
\begin{aligned}
   \partial_t\frac{\rho}{2}\fint\norm{\bmu}^2 &= -2\mu\fint\norm{\varepsilon(\bmu)}^2 - \frac{\rho}{2}\bint\bmu\cdot\bmn\norm{\bmu}^2\\
                                              &= -\qty(2\mu\fint\norm{\varepsilon(\bmu)}^2 + \frac{\rho}{2}\bint\bmu\cdot\bmn\norm{\bmu}^2)\\
                                              &= -\qty(2\mu\fint\norm{\varepsilon(\bmu)}^2 + \frac{\rho}{2}\bint\abs{\bmu\cdot\bmn}_+\norm{\bmu}^2 - \frac{\rho}{2}\bint\abs{\bmu\cdot\bmn}_-\norm{\bmu}^2)
\end{aligned}
\]
    \end{frame}
    
    \begin{frame}{Backflow interaction}

Want the term \(B := \frac{\rho}{2}\bint\abs{\bmu\cdot\bmn}_- \bmu \cdot \bmv\) to be stabilised, so that when we test with \(\bmv = \bmu\), we get that the energy decreases, i.e. dissipation.\\
We thus need to add one of the stabilisations, and look at \(-B+S\).
    \end{frame}
    
    \begin{frame}{Stability Criterion}
    We take a very conservative approach and look only at the added stability terms to counter the instability
        \begin{block}{Stability Criterion}
            For stability, we require that the eigenvalues of \(-B+S\) in discretised form are non-negative
        \end{block}
    \end{frame}
    \subsection{Reduction of Eigenvalue Problem}
    \begin{frame}{Checking the stability criterion}
    \begin{block}{What are the problems with checking the criterion?}
        Computationally Expensive - may have to calculate the eigenvalues of a few thousand by few thousand matrix at every iteration
    \end{block}
    Idea: Don't have to consider the entire matrix
    \end{frame}
    \begin{frame}{Creating submatrix}
        Note: \(B\) is only non-zero in the presence of backflow.\\
        \(\implies B\) is exceptionally sparse when discretised.\\
        \(\implies\) Only a few small non-zero blocks\\
        \(\therefore\) We could then rather extract the block matrices of \(-B+S\) where the discretisation of \(B\) is non-zero.
        %Note that B is only non-zero in the presence of backflow so in its discretised form it is exceptionally %sparse, indeed most of the matrix is zeros besides a few small blocks.  %rewite more compact
    \end{frame}