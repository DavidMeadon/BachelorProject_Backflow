%% Useful packages

%% Language and font encodings
\usepackage[english]{babel}
\usepackage[utf8]{inputenc}
\usepackage[T1]{fontenc}
\usepackage{amsfonts}

%% Sets page size and margins
% \usepackage[a4paper,top=3cm,bottom=2cm,left=3cm,right=3cm,marginparwidth=1.75cm]{geometry}

%% Math
\usepackage{amsmath} %
\usepackage{amssymb}
\usepackage{amsthm}
\usepackage{mathtools}
\usepackage{physics}

%% Graphics
\usepackage{graphicx}
\usepackage{xcolor}

%% Format
\usepackage[colorinlistoftodos]{todonotes}
\usepackage[colorlinks=true, allcolors=blue]{hyperref}
\usepackage{enumerate}
% \usepackage[backend=biber,style=ieee,url=true,doi=true,dashed=false,sorting=nyt]{biblatex}
% \addbibresource{citations.bib}
\usepackage{comment}

%% Code
\usepackage{minted}


%%%%%%--FOR ABS AND NORM--%%%%%%
% \DeclarePairedDelimiter\abs{\lvert}{\rvert}
% \DeclarePairedDelimiter\norm{\lVert}{\rVert}

% \makeatletter
% \let\oldabs\abs
% \def\abs{\@ifstar{\oldabs}{\oldabs*}}
% %
% \let\oldnorm\norm
% \def\norm{\@ifstar{\oldnorm}{\oldnorm*}}
% \makeatother
%%%%%%%%%%%%%%%%%%%%%%%%%%%%%%%%%

%%%%%%--MINI MATH MODE--%%%%%%
\newcommand{\mathm}[1]{\(#1\)}
%%%%%%%%%%%%%%%%%%%%%%%%%%%%%%

%%%%%%%%%%%%%%%%%%%%%%%%--This is for simpler integrals--%%%%%%%%%%%%%%%%%%%%%%%
\usepackage{xparse}

\newcommand\dd{\mathrm{d}}

\ExplSyntaxOn
\NewDocumentCommand \Int { s o m o }
  {%
    \IfNoValueTF{ #2 }
      { \int }
      {
        \fiziks_int:nn { #1 } { #2 }
      }
    #3
    \IfNoValueF { #4 } { \fiziks_int_dx:n { #4 } }
  }

\seq_new:N \l__fiziks_int_args_seq
\tl_new:N \l__fiziks_int_ast_tl

\cs_new_protected:Npn \fiziks_int:nn #1 #2
  {
    \seq_set_split:Nnn \l__fiziks_int_args_seq { ; } { #2 }
    \seq_map_inline:Nn \l__fiziks_int_args_seq 
      {
        \tl_if_in:nnTF { ##1 } { * }
          {% * case
            \tl_set:Nn \l__fiziks_int_ast_tl { ##1 }
            \tl_remove_once:Nn \l__fiziks_int_ast_tl { * }
            \fiziks_int_inner:NnV \oint { #1 } \l__fiziks_int_ast_tl
          }
          {% no * case
            \fiziks_int_inner:Nnn \int { #1 } { ##1 }
          }
      }
  }
\cs_new_protected:Npn \fiziks_int_inner:Nnn #1 #2 #3
  {
    #1
    \tl_if_blank:nF { #3 } 
      {
        \IfBooleanT { #2 } { \limits }
        \fiziks_int_inner_inner:Nx \sb { \clist_item:nn { #3 } { 1 } }
        \fiziks_int_inner_inner:Nx \sp { \clist_item:nn { #3 } { 2 } }
      }
  }
\cs_generate_variant:Nn \fiziks_int_inner:Nnn { NnV }
\cs_new:Npn \fiziks_int_inner_inner:Nn #1 #2
  {
    \tl_if_blank:nF { #2 } { #1 { #2 } }
  }
\cs_generate_variant:Nn \fiziks_int_inner_inner:Nn { Nx }

\cs_new_protected:Npn \fiziks_int_dx:n #1
  {
    \seq_set_split:Nnn \l__fiziks_int_args_seq { ; } { #1 }
    \seq_map_inline:Nn \l__fiziks_int_args_seq
      {
        \,\dd##1
      }
  }
\ExplSyntaxOff
%%      Can be used as \Int[0,T;0,a]{f^2}[t;x]    
%%%%%%%%%%%%%%%%%%%%%%%%%%%%%%%%%%%%%%%%%%%%%%%%%%%%%%%%%%%%%%%%%%%%%%%%%%%%%%%%

\DeclareRobustCommand{\orderof}{\ensuremath{\mathcal{O}}}